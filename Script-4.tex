% Options for packages loaded elsewhere
\PassOptionsToPackage{unicode}{hyperref}
\PassOptionsToPackage{hyphens}{url}
%
\documentclass[
]{article}
\usepackage{amsmath,amssymb}
\usepackage{iftex}
\ifPDFTeX
  \usepackage[T1]{fontenc}
  \usepackage[utf8]{inputenc}
  \usepackage{textcomp} % provide euro and other symbols
\else % if luatex or xetex
  \usepackage{unicode-math} % this also loads fontspec
  \defaultfontfeatures{Scale=MatchLowercase}
  \defaultfontfeatures[\rmfamily]{Ligatures=TeX,Scale=1}
\fi
\usepackage{lmodern}
\ifPDFTeX\else
  % xetex/luatex font selection
\fi
% Use upquote if available, for straight quotes in verbatim environments
\IfFileExists{upquote.sty}{\usepackage{upquote}}{}
\IfFileExists{microtype.sty}{% use microtype if available
  \usepackage[]{microtype}
  \UseMicrotypeSet[protrusion]{basicmath} % disable protrusion for tt fonts
}{}
\makeatletter
\@ifundefined{KOMAClassName}{% if non-KOMA class
  \IfFileExists{parskip.sty}{%
    \usepackage{parskip}
  }{% else
    \setlength{\parindent}{0pt}
    \setlength{\parskip}{6pt plus 2pt minus 1pt}}
}{% if KOMA class
  \KOMAoptions{parskip=half}}
\makeatother
\usepackage{xcolor}
\usepackage[margin=1in]{geometry}
\usepackage{color}
\usepackage{fancyvrb}
\newcommand{\VerbBar}{|}
\newcommand{\VERB}{\Verb[commandchars=\\\{\}]}
\DefineVerbatimEnvironment{Highlighting}{Verbatim}{commandchars=\\\{\}}
% Add ',fontsize=\small' for more characters per line
\usepackage{framed}
\definecolor{shadecolor}{RGB}{248,248,248}
\newenvironment{Shaded}{\begin{snugshade}}{\end{snugshade}}
\newcommand{\AlertTok}[1]{\textcolor[rgb]{0.94,0.16,0.16}{#1}}
\newcommand{\AnnotationTok}[1]{\textcolor[rgb]{0.56,0.35,0.01}{\textbf{\textit{#1}}}}
\newcommand{\AttributeTok}[1]{\textcolor[rgb]{0.13,0.29,0.53}{#1}}
\newcommand{\BaseNTok}[1]{\textcolor[rgb]{0.00,0.00,0.81}{#1}}
\newcommand{\BuiltInTok}[1]{#1}
\newcommand{\CharTok}[1]{\textcolor[rgb]{0.31,0.60,0.02}{#1}}
\newcommand{\CommentTok}[1]{\textcolor[rgb]{0.56,0.35,0.01}{\textit{#1}}}
\newcommand{\CommentVarTok}[1]{\textcolor[rgb]{0.56,0.35,0.01}{\textbf{\textit{#1}}}}
\newcommand{\ConstantTok}[1]{\textcolor[rgb]{0.56,0.35,0.01}{#1}}
\newcommand{\ControlFlowTok}[1]{\textcolor[rgb]{0.13,0.29,0.53}{\textbf{#1}}}
\newcommand{\DataTypeTok}[1]{\textcolor[rgb]{0.13,0.29,0.53}{#1}}
\newcommand{\DecValTok}[1]{\textcolor[rgb]{0.00,0.00,0.81}{#1}}
\newcommand{\DocumentationTok}[1]{\textcolor[rgb]{0.56,0.35,0.01}{\textbf{\textit{#1}}}}
\newcommand{\ErrorTok}[1]{\textcolor[rgb]{0.64,0.00,0.00}{\textbf{#1}}}
\newcommand{\ExtensionTok}[1]{#1}
\newcommand{\FloatTok}[1]{\textcolor[rgb]{0.00,0.00,0.81}{#1}}
\newcommand{\FunctionTok}[1]{\textcolor[rgb]{0.13,0.29,0.53}{\textbf{#1}}}
\newcommand{\ImportTok}[1]{#1}
\newcommand{\InformationTok}[1]{\textcolor[rgb]{0.56,0.35,0.01}{\textbf{\textit{#1}}}}
\newcommand{\KeywordTok}[1]{\textcolor[rgb]{0.13,0.29,0.53}{\textbf{#1}}}
\newcommand{\NormalTok}[1]{#1}
\newcommand{\OperatorTok}[1]{\textcolor[rgb]{0.81,0.36,0.00}{\textbf{#1}}}
\newcommand{\OtherTok}[1]{\textcolor[rgb]{0.56,0.35,0.01}{#1}}
\newcommand{\PreprocessorTok}[1]{\textcolor[rgb]{0.56,0.35,0.01}{\textit{#1}}}
\newcommand{\RegionMarkerTok}[1]{#1}
\newcommand{\SpecialCharTok}[1]{\textcolor[rgb]{0.81,0.36,0.00}{\textbf{#1}}}
\newcommand{\SpecialStringTok}[1]{\textcolor[rgb]{0.31,0.60,0.02}{#1}}
\newcommand{\StringTok}[1]{\textcolor[rgb]{0.31,0.60,0.02}{#1}}
\newcommand{\VariableTok}[1]{\textcolor[rgb]{0.00,0.00,0.00}{#1}}
\newcommand{\VerbatimStringTok}[1]{\textcolor[rgb]{0.31,0.60,0.02}{#1}}
\newcommand{\WarningTok}[1]{\textcolor[rgb]{0.56,0.35,0.01}{\textbf{\textit{#1}}}}
\usepackage{graphicx}
\makeatletter
\newsavebox\pandoc@box
\newcommand*\pandocbounded[1]{% scales image to fit in text height/width
  \sbox\pandoc@box{#1}%
  \Gscale@div\@tempa{\textheight}{\dimexpr\ht\pandoc@box+\dp\pandoc@box\relax}%
  \Gscale@div\@tempb{\linewidth}{\wd\pandoc@box}%
  \ifdim\@tempb\p@<\@tempa\p@\let\@tempa\@tempb\fi% select the smaller of both
  \ifdim\@tempa\p@<\p@\scalebox{\@tempa}{\usebox\pandoc@box}%
  \else\usebox{\pandoc@box}%
  \fi%
}
% Set default figure placement to htbp
\def\fps@figure{htbp}
\makeatother
\setlength{\emergencystretch}{3em} % prevent overfull lines
\providecommand{\tightlist}{%
  \setlength{\itemsep}{0pt}\setlength{\parskip}{0pt}}
\setcounter{secnumdepth}{-\maxdimen} % remove section numbering
\usepackage{bookmark}
\IfFileExists{xurl.sty}{\usepackage{xurl}}{} % add URL line breaks if available
\urlstyle{same}
\hypersetup{
  pdftitle={Script-4.R},
  pdfauthor={Usuario},
  hidelinks,
  pdfcreator={LaTeX via pandoc}}

\title{Script-4.R}
\author{Usuario}
\date{2025-08-28}

\begin{document}
\maketitle

\begin{Shaded}
\begin{Highlighting}[]
\CommentTok{\# Script 4}
\CommentTok{\# 28/08/2025}
\CommentTok{\# Javier Francisco Santos}

\CommentTok{\# Importar datos}
\NormalTok{calidad }\OtherTok{\textless{}{-}} \FunctionTok{read.csv}\NormalTok{(}\StringTok{"calidadplanta.csv"}\NormalTok{, }\AttributeTok{header =} \ConstantTok{TRUE}\NormalTok{)}
\FunctionTok{View}\NormalTok{(calidad)}

\CommentTok{\# Asegurar que Tratamiento es un factor}
\NormalTok{calidad}\SpecialCharTok{$}\NormalTok{Tratamiento }\OtherTok{\textless{}{-}} \FunctionTok{as.factor}\NormalTok{(calidad}\SpecialCharTok{$}\NormalTok{Tratamiento)}
\FunctionTok{class}\NormalTok{(calidad)}
\end{Highlighting}
\end{Shaded}

\begin{verbatim}
## [1] "data.frame"
\end{verbatim}

\begin{Shaded}
\begin{Highlighting}[]
\FunctionTok{summary}\NormalTok{(calidad)}
\end{Highlighting}
\end{Shaded}

\begin{verbatim}
##      planta            IE         Tratamiento
##  Min.   : 1.00   Min.   :0.5500   Ctrl:21    
##  1st Qu.:11.25   1st Qu.:0.7025   Fert:21    
##  Median :21.50   Median :0.7950              
##  Mean   :21.50   Mean   :0.8371              
##  3rd Qu.:31.75   3rd Qu.:0.9375              
##  Max.   :42.00   Max.   :1.1600
\end{verbatim}

\begin{Shaded}
\begin{Highlighting}[]
\CommentTok{\# Medidas descriptivas}
\FunctionTok{mean}\NormalTok{(calidad}\SpecialCharTok{$}\NormalTok{IE)}
\end{Highlighting}
\end{Shaded}

\begin{verbatim}
## [1] 0.8371429
\end{verbatim}

\begin{Shaded}
\begin{Highlighting}[]
\FunctionTok{tapply}\NormalTok{(calidad}\SpecialCharTok{$}\NormalTok{IE, calidad}\SpecialCharTok{$}\NormalTok{Tratamiento, mean)}
\end{Highlighting}
\end{Shaded}

\begin{verbatim}
##      Ctrl      Fert 
## 0.7676190 0.9066667
\end{verbatim}

\begin{Shaded}
\begin{Highlighting}[]
\FunctionTok{tapply}\NormalTok{(calidad}\SpecialCharTok{$}\NormalTok{IE, calidad}\SpecialCharTok{$}\NormalTok{Tratamiento, sd)}
\end{Highlighting}
\end{Shaded}

\begin{verbatim}
##      Ctrl      Fert 
## 0.1153215 0.1799537
\end{verbatim}

\begin{Shaded}
\begin{Highlighting}[]
\FunctionTok{tapply}\NormalTok{(calidad}\SpecialCharTok{$}\NormalTok{IE, calidad}\SpecialCharTok{$}\NormalTok{Tratamiento, var)}
\end{Highlighting}
\end{Shaded}

\begin{verbatim}
##       Ctrl       Fert 
## 0.01329905 0.03238333
\end{verbatim}

\begin{Shaded}
\begin{Highlighting}[]
\CommentTok{\# Gráficos de caja}
\NormalTok{colores }\OtherTok{\textless{}{-}} \FunctionTok{c}\NormalTok{(}\StringTok{"green"}\NormalTok{, }\StringTok{"red"}\NormalTok{)}

\FunctionTok{boxplot}\NormalTok{(calidad}\SpecialCharTok{$}\NormalTok{IE }\SpecialCharTok{\textasciitilde{}}\NormalTok{ calidad}\SpecialCharTok{$}\NormalTok{Tratamiento,}
        \AttributeTok{col =}\NormalTok{ colores,}
        \AttributeTok{xlab =} \StringTok{"Tratamiento aplicado"}\NormalTok{,}
        \AttributeTok{ylab =} \StringTok{"Índice de Eficiencia (IE)"}\NormalTok{,}
        \AttributeTok{main =} \StringTok{"Calidad de plantas"}\NormalTok{)}
\end{Highlighting}
\end{Shaded}

\pandocbounded{\includegraphics[keepaspectratio]{Script-4_files/figure-latex/unnamed-chunk-1-1.pdf}}

\begin{Shaded}
\begin{Highlighting}[]
\FunctionTok{boxplot}\NormalTok{(calidad}\SpecialCharTok{$}\NormalTok{IE }\SpecialCharTok{\textasciitilde{}}\NormalTok{ calidad}\SpecialCharTok{$}\NormalTok{Tratamiento,}
        \AttributeTok{col =}\NormalTok{ colores,}
        \AttributeTok{xlab =} \StringTok{"Tratamiento aplicado"}\NormalTok{,}
        \AttributeTok{ylab =} \StringTok{"Índice de Eficiencia (IE)"}\NormalTok{,}
        \AttributeTok{ylim =} \FunctionTok{c}\NormalTok{(}\FloatTok{0.4}\NormalTok{, }\FloatTok{1.2}\NormalTok{),}
        \AttributeTok{main =} \StringTok{"Calidad de plantas"}\NormalTok{)}
\end{Highlighting}
\end{Shaded}

\pandocbounded{\includegraphics[keepaspectratio]{Script-4_files/figure-latex/unnamed-chunk-1-2.pdf}}

\begin{Shaded}
\begin{Highlighting}[]
\CommentTok{\# Subconjuntos por tratamiento}
\NormalTok{df\_ctrl }\OtherTok{\textless{}{-}} \FunctionTok{subset}\NormalTok{(calidad, Tratamiento }\SpecialCharTok{==} \StringTok{"ctrl"}\NormalTok{)}
\NormalTok{df\_fert }\OtherTok{\textless{}{-}} \FunctionTok{subset}\NormalTok{(calidad, Tratamiento }\SpecialCharTok{==} \StringTok{"fert"}\NormalTok{)  }\CommentTok{\# Asegúrate de que "fert" sea el otro tratamiento correcto}

\CommentTok{\# Crear los subconjuntos}
\NormalTok{df\_ctrl }\OtherTok{\textless{}{-}} \FunctionTok{subset}\NormalTok{(calidad, Tratamiento }\SpecialCharTok{==} \StringTok{"Ctrl"}\NormalTok{)}
\NormalTok{df\_fert }\OtherTok{\textless{}{-}} \FunctionTok{subset}\NormalTok{(calidad, Tratamiento }\SpecialCharTok{==} \StringTok{"Fert"}\NormalTok{)  }\CommentTok{\# Asegúrate del nombre exacto}

\CommentTok{\# QQ{-}plots para comprobar normalidad}
\FunctionTok{par}\NormalTok{(}\AttributeTok{mfrow =} \FunctionTok{c}\NormalTok{(}\DecValTok{1}\NormalTok{, }\DecValTok{2}\NormalTok{))}

\FunctionTok{qqnorm}\NormalTok{(}\FunctionTok{na.omit}\NormalTok{(df\_ctrl}\SpecialCharTok{$}\NormalTok{IE)); }\FunctionTok{qqline}\NormalTok{(}\FunctionTok{na.omit}\NormalTok{(df\_ctrl}\SpecialCharTok{$}\NormalTok{IE))}
\FunctionTok{qqnorm}\NormalTok{(}\FunctionTok{na.omit}\NormalTok{(df\_fert}\SpecialCharTok{$}\NormalTok{IE)); }\FunctionTok{qqline}\NormalTok{(}\FunctionTok{na.omit}\NormalTok{(df\_fert}\SpecialCharTok{$}\NormalTok{IE))}
\end{Highlighting}
\end{Shaded}

\pandocbounded{\includegraphics[keepaspectratio]{Script-4_files/figure-latex/unnamed-chunk-1-3.pdf}}

\begin{Shaded}
\begin{Highlighting}[]
\FunctionTok{par}\NormalTok{(}\AttributeTok{mfrow =} \FunctionTok{c}\NormalTok{(}\DecValTok{1}\NormalTok{, }\DecValTok{1}\NormalTok{))}

\CommentTok{\# Prueba de normalidad}
\FunctionTok{shapiro.test}\NormalTok{(df\_ctrl}\SpecialCharTok{$}\NormalTok{IE)}
\end{Highlighting}
\end{Shaded}

\begin{verbatim}
## 
##  Shapiro-Wilk normality test
## 
## data:  df_ctrl$IE
## W = 0.9532, p-value = 0.3908
\end{verbatim}

\begin{Shaded}
\begin{Highlighting}[]
\FunctionTok{shapiro.test}\NormalTok{(df\_fert}\SpecialCharTok{$}\NormalTok{IE)}
\end{Highlighting}
\end{Shaded}

\begin{verbatim}
## 
##  Shapiro-Wilk normality test
## 
## data:  df_fert$IE
## W = 0.95339, p-value = 0.3941
\end{verbatim}

\begin{Shaded}
\begin{Highlighting}[]
\CommentTok{\# Prueba de homogeneidad de varianzas}
\FunctionTok{var.test}\NormalTok{(calidad}\SpecialCharTok{$}\NormalTok{IE }\SpecialCharTok{\textasciitilde{}}\NormalTok{ calidad}\SpecialCharTok{$}\NormalTok{Tratamiento)}
\end{Highlighting}
\end{Shaded}

\begin{verbatim}
## 
##  F test to compare two variances
## 
## data:  calidad$IE by calidad$Tratamiento
## F = 0.41068, num df = 20, denom df = 20, p-value = 0.05304
## alternative hypothesis: true ratio of variances is not equal to 1
## 95 percent confidence interval:
##  0.1666376 1.0121038
## sample estimates:
## ratio of variances 
##          0.4106757
\end{verbatim}

\begin{Shaded}
\begin{Highlighting}[]
\CommentTok{\# Prueba t de Student (dos colas)}
\FunctionTok{t.test}\NormalTok{(calidad}\SpecialCharTok{$}\NormalTok{IE }\SpecialCharTok{\textasciitilde{}}\NormalTok{ calidad}\SpecialCharTok{$}\NormalTok{Tratamiento,}
       \AttributeTok{alternative =} \StringTok{"two.sided"}\NormalTok{,}
       \AttributeTok{var.equal =} \ConstantTok{TRUE}\NormalTok{)}
\end{Highlighting}
\end{Shaded}

\begin{verbatim}
## 
##  Two Sample t-test
## 
## data:  calidad$IE by calidad$Tratamiento
## t = -2.9813, df = 40, p-value = 0.004868
## alternative hypothesis: true difference in means between group Ctrl and group Fert is not equal to 0
## 95 percent confidence interval:
##  -0.23331192 -0.04478332
## sample estimates:
## mean in group Ctrl mean in group Fert 
##          0.7676190          0.9066667
\end{verbatim}

\begin{Shaded}
\begin{Highlighting}[]
\CommentTok{\# Ejemplo erróneo (una cola: mayor)}
\FunctionTok{t.test}\NormalTok{(calidad}\SpecialCharTok{$}\NormalTok{IE }\SpecialCharTok{\textasciitilde{}}\NormalTok{ calidad}\SpecialCharTok{$}\NormalTok{Tratamiento,}
       \AttributeTok{alternative =} \StringTok{"greater"}\NormalTok{,}
       \AttributeTok{var.equal =} \ConstantTok{TRUE}\NormalTok{)}
\end{Highlighting}
\end{Shaded}

\begin{verbatim}
## 
##  Two Sample t-test
## 
## data:  calidad$IE by calidad$Tratamiento
## t = -2.9813, df = 40, p-value = 0.9976
## alternative hypothesis: true difference in means between group Ctrl and group Fert is greater than 0
## 95 percent confidence interval:
##  -0.2175835        Inf
## sample estimates:
## mean in group Ctrl mean in group Fert 
##          0.7676190          0.9066667
\end{verbatim}

\begin{Shaded}
\begin{Highlighting}[]
\CommentTok{\# Cálculo del tamaño del efecto de Cohen\textquotesingle{}s d}
\NormalTok{cohens\_efecto }\OtherTok{\textless{}{-}} \ControlFlowTok{function}\NormalTok{(x, y) \{}
\NormalTok{  n1 }\OtherTok{\textless{}{-}} \FunctionTok{length}\NormalTok{(x); n2 }\OtherTok{\textless{}{-}} \FunctionTok{length}\NormalTok{(y)}
\NormalTok{  s1 }\OtherTok{\textless{}{-}} \FunctionTok{sd}\NormalTok{(x); s2 }\OtherTok{\textless{}{-}} \FunctionTok{sd}\NormalTok{(y)}
\NormalTok{  sp }\OtherTok{\textless{}{-}} \FunctionTok{sqrt}\NormalTok{(((n1 }\SpecialCharTok{{-}} \DecValTok{1}\NormalTok{) }\SpecialCharTok{*}\NormalTok{ s1}\SpecialCharTok{\^{}}\DecValTok{2} \SpecialCharTok{+}\NormalTok{ (n2 }\SpecialCharTok{{-}} \DecValTok{1}\NormalTok{) }\SpecialCharTok{*}\NormalTok{ s2}\SpecialCharTok{\^{}}\DecValTok{2}\NormalTok{) }\SpecialCharTok{/}\NormalTok{ (n1 }\SpecialCharTok{+}\NormalTok{ n2 }\SpecialCharTok{{-}} \DecValTok{2}\NormalTok{))}
\NormalTok{  (}\FunctionTok{mean}\NormalTok{(x) }\SpecialCharTok{{-}} \FunctionTok{mean}\NormalTok{(y)) }\SpecialCharTok{/}\NormalTok{ sp}
\NormalTok{\}}

\NormalTok{d1\_cal }\OtherTok{\textless{}{-}} \FunctionTok{cohens\_efecto}\NormalTok{(df\_ctrl}\SpecialCharTok{$}\NormalTok{IE, df\_fert}\SpecialCharTok{$}\NormalTok{IE)}
\NormalTok{d1\_cal}
\end{Highlighting}
\end{Shaded}

\begin{verbatim}
## [1] -0.9200347
\end{verbatim}

\end{document}
